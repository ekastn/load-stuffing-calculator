\section{Conclusion}

This research achieved its three primary objectives in developing a web-based container loading planning system. First, a layered system architecture was successfully designed and implemented, effectively separating data management from the 3D bin packing calculation engine while enabling seamless integration through a RESTful API. The architecture supports role-based access control for administrators, planners, and operators with distinct permission levels. Second, interactive 3D visualization with step-by-step playback was developed using WebGL-based rendering, providing actionable loading guidance that bridges algorithmic optimization with physical execution requirements. The visualization enables operators to follow item placement sequences chronologically without requiring specialized desktop software installation. Third, system effectiveness was validated through five heterogeneous item scenarios with varying complexity levels, demonstrating practical applicability for offline warehouse operational planning.

Experimental evaluation revealed several key quantitative findings. The system achieved 55.26\% volume utilization in the most complex scenario featuring four distinct product types and 300 item units, with placement success rates (fill rate) consistently reaching 100\% across all test scenarios. The Bigger First sorting strategy combined with static stability constraints produced optimal space filling, outperforming Smaller First by 43\% in volume utilization. This performance aligns with literature findings where similar deterministic approaches achieve 60--85\% utilization depending on item heterogeneity. Stability checking added minimal computational overhead of approximately 8\% while ensuring physical feasibility through 75\% minimum base area support requirements. Computation time exhibited quadratic growth patterns with item population, reaching approximately 38 seconds for 300 units---a duration well within tolerance limits for offline planning operations where plans are prepared before physical loading begins.

Several limitations should be acknowledged. The current implementation employs a deterministic heuristic algorithm without solution space exploration capabilities that metaheuristic approaches such as genetic algorithms or simulated annealing would provide, potentially limiting achievable utilization rates. Test scenarios utilized standard rectangular items, excluding irregular or deformable shapes \cite{Zuo2022} that occur in certain logistics contexts such as fresh food delivery. Furthermore, while the system was tested with realistic heterogeneous item configurations, no formal user study with actual warehouse operators was conducted to validate usability and practical effectiveness in field conditions. Weight constraints, although implemented, were not tested at capacity limits across scenarios.

Future research can extend this work in multiple directions. Integration with metaheuristic optimization algorithms such as genetic algorithms or simulated annealing \cite{Tsao2024} could improve space utilization beyond deterministic heuristic limits through broader solution space exploration. Deep Reinforcement Learning approaches offer potential for adaptive online packing decisions that respond to real-time item arrivals, as demonstrated by recent advances \cite{Zhao2022, Murdivien2023} achieving up to 10\% utilization improvements over conventional methods. IoT sensor integration would enable automatic item dimension capture and real-time tracking during loading operations, facilitating validation between planned and actual placements. A formal user study with warehouse operators would provide empirical validation of the visualization interface effectiveness and identify areas for usability improvement. Finally, mobile interface development would enhance field accessibility for operators who require step-by-step guidance during physical loading execution.
