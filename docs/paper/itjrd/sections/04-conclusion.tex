\section{Conclusion}

This research has successfully completed the development and evaluation of a web-based Load \& Stuffing Calculator system designed to assist heterogeneous cargo loading optimization. Through layered architecture separating data management and calculation engines, the system proves capable of effectively integrating 3D bin packing algorithms into a web-based environment without significant operational constraints. Testing results demonstrate that the system can handle loading scenarios of up to 300 item units with placement success rates (fill rate) consistently reaching 100\% across all test scenarios.

From algorithm performance perspective, the use of Bigger First strategy combined with static stability constraints and gravity simulation produces optimal space filling. The system recorded 55.26\% volume utilization in the most complex cargo scenario, a competitive figure meeting industry efficiency standards for arranging goods with highly variable dimensions and types. Although computation time exhibits quadratic growth patterns as item population increases, the maximum duration of approximately 38 seconds for calculating 300 units remains well within tolerance limits for offline warehouse operational planning needs.

Beyond quantitative achievements, interactive three-dimensional visualization implementation provides important contributions in bridging theoretical calculations with physical field execution. The step-by-step playback feature enables operators to view item placement sequences chronologically while considering stack safety aspects through minimum 75\% base area support. Overall, the integration between precise algorithmic calculations and intuitive visual interfaces makes this system a practical decision support tool for improving logistics process efficiency. Future research can be directed toward developing more dynamic packing strategies for various container types to continuously improve cargo density.
