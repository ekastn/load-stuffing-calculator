\section{Results and Discussion}
\label{sec:results}

Testing was conducted based on the scenarios defined in the methodology. Results address space utilization, computation time scalability, algorithm variant performance, and visual validation.

\subsection{Experimental Setup}

Testing was conducted on a system with Intel Core i5-6440HQ @ 2.60GHz processor, 16 GB RAM, and Linux operating system. The application used Python 3.11 for the packing service and Go 1.21 for the backend API. Each scenario was executed 25 times to obtain mean values and standard deviations.

The container used throughout testing was a 40-foot High Cube type with internal dimensions of 12.032 $\times$ 2.352 $\times$ 2.698 meters (volume 76.35 m\textsuperscript{3}) and maximum load capacity of 26,460 kg. Four product types with varying dimensions were defined to simulate item heterogeneity: Euro Pallet (1200$\times$800$\times$144 mm), Large Crate (1000$\times$600$\times$500 mm), Medium Box (600$\times$400$\times$400 mm), and Small Box (400$\times$300$\times$200 mm). Each scenario uses a subset of these product types as specified in Table \ref{tab:scenarios}.

\subsection{Performance Results}

Table \ref{tab:results} presents testing results for all five scenarios using the baseline algorithm configuration (Bigger First strategy with stability checking enabled). Measured metrics include volume utilization, weight utilization, fill rate, and computation time.

\begin{table}[h]
	\centering
	\caption{Loading Algorithm Performance Testing Results}
	\label{tab:results}
	\small
	\begin{tabular}{lrrrrr}
		\toprule
		\textbf{Scenario} & \textbf{Items} & \textbf{Vol. Util. (\%)} & \textbf{Weight Util. (\%)} & \textbf{Fill Rate (\%)} & \textbf{Time (ms)} \\
		\midrule
		S1                & 50             & 6.29 $\pm$ 0.00          & 2.83 $\pm$ 0.00            & 100.00                  & 320 $\pm$ 6        \\
		S2                & 100            & 8.80 $\pm$ 0.00          & 4.16 $\pm$ 0.00            & 100.00                  & 1,684 $\pm$ 131    \\
		S3                & 150            & 24.83 $\pm$ 0.00         & 11.15 $\pm$ 0.00           & 100.00                  & 5,213 $\pm$ 169    \\
		S4                & 200            & 35.45 $\pm$ 0.00         & 16.25 $\pm$ 0.00           & 100.00                  & 10,826 $\pm$ 168   \\
		S5                & 300            & 55.26 $\pm$ 0.00         & 25.32 $\pm$ 0.00           & 100.00                  & 38,343 $\pm$ 1,209 \\
		\bottomrule
	\end{tabular}
\end{table}

The standard deviation of 0.00 for utilization metrics reflects the deterministic nature of the algorithm, where identical inputs produce identical placement results. Only computation time exhibits variance due to system-level factors such as process scheduling and memory allocation.

The algorithm successfully placed all items (100\% fill rate) across all scenarios, indicating that the 40-foot High Cube container capacity is adequate for the tested item configurations. The relatively low utilization values in scenarios S1 and S2 reflect the test design rather than algorithm inefficiency; these scenarios intentionally use fewer items than required to fill the container, allowing assessment of algorithm behavior across varying load levels. Volume utilization increases proportionally with item count, from 6.29\% in S1 to 55.26\% in S5. Weight utilization values consistently lower than volume utilization indicate that the configuration is volume-constrained rather than weight-constrained.

Figure \ref{fig:utilization} visualizes the comparison of volume and weight utilization for each scenario. Volume utilization consistently exceeds weight utilization with approximately a 2:1 ratio, confirming the volume-constrained characteristics of the test configuration.

\begin{figure}[H]
	\centering
	\includegraphics[width=0.9\textwidth]{figures/utilization_comparison.pdf}
	\caption{Comparison of volume and weight utilization in each scenario}
	\label{fig:utilization}
\end{figure}

\subsection{Computation Time Analysis}

An important consideration in 3D-BPP evaluation is computation time scalability with increasing item counts. Table \ref{tab:scalability} presents detailed computation time statistics, including minimum, average, and maximum values for each scenario.

\begin{table}[h]
	\centering
	\caption{Computation Time Scalability Analysis}
	\label{tab:scalability}
	\small
	\begin{tabular}{lrrrr}
		\toprule
		\textbf{Scenario} & \textbf{Items} & \textbf{Min (ms)} & \textbf{Avg (ms)} & \textbf{Max (ms)} \\
		\midrule
		S1                & 50             & 311               & 320               & 325               \\
		S2                & 100            & 1,489             & 1,684             & 1,791             \\
		S3                & 150            & 5,012             & 5,213             & 5,391             \\
		S4                & 200            & 10,544            & 10,826            & 10,972            \\
		S5                & 300            & 36,334            & 38,343            & 39,288            \\
		\bottomrule
	\end{tabular}
\end{table}

Figure \ref{fig:scalability} illustrates the relationship between item count and computation time. Computation time exhibits quadratic growth ($O(n^2)$), consistent with heuristic-based bin packing algorithm characteristics \cite{Ma2025}. For each item placement, the algorithm evaluates potential collisions with all previously positioned items, resulting in quadratic complexity.

\begin{figure}[H]
	\centering
	\includegraphics[width=0.9\textwidth]{figures/computation_time.pdf}
	\caption{Computation time scalability with respect to item count}
	\label{fig:scalability}
\end{figure}

Although computation time in scenario S5 (300 items) reaches an average of 38 seconds, this value remains acceptable for offline loading planning applications. In logistics operational contexts, loading planning is typically performed before physical loading commences, making wait times under one minute acceptable for operational requirements \cite{Tresca2022}.

\subsection{Algorithm Variant Comparison}

To assess the contribution of each algorithm component, comparative testing was performed with three configuration variants on scenario S3 (150 items). Table \ref{tab:variants} presents comparison results between the baseline configuration (Bigger First with Stability), a variant without stability checking, and a variant using the Smaller First strategy.

\begin{table}[h]
	\centering
	\caption{Algorithm Variant Comparison in Scenario S3 (150 items)}
	\label{tab:variants}
	\small
	\begin{tabular}{lrrrr}
		\toprule
		\textbf{Variant}            & \textbf{Vol. Util. (\%)} & \textbf{Fill Rate (\%)} & \textbf{Time (ms)} & \textbf{Items Packed} \\
		\midrule
		Bigger First + Stability    & 24.83                    & 100.00                  & 5,386              & 150                   \\
		Bigger First (No Stability) & 24.83                    & 100.00                  & 4,960              & 150                   \\
		Smaller First + Stability   & 17.37                    & 87.33                   & 32,886             & 131                   \\
		\bottomrule
	\end{tabular}
\end{table}

The comparison reveals that the Bigger First strategy substantially outperforms Smaller First, yielding 43\% relatively higher volume utilization (24.83\% vs. 17.37\%) and achieving complete placement compared to only 87.33\% fill rate. This confirms the heuristic principle that placing large items first leaves small gaps fillable by smaller items, whereas the reverse approach creates suboptimal fragmentation \cite{Ma2025}. The Smaller First variant also requires 6.1 times longer execution (32,886 ms vs. 5,386 ms) due to increased failed placement attempts when small items occupy spaces more suitable for large items.

Disabling stability checking saves only approximately 8\% computation time (4,960 ms vs. 5,386 ms) without affecting utilization results. This indicates that stability checking overhead is relatively small and worth retaining to ensure physical loading safety.

\subsection{Validation Results}

Placement feasibility was validated across all five scenarios comprising 800 total item placements. Three geometric validation criteria were evaluated: non-overlapping (no item intersections), boundary containment (all items within container walls), and stability compliance (minimum 75\% base support area for elevated items). Validation employed computational geometric checks during the packing process combined with visual verification through the implemented 3D visualization interface (Figure \ref{fig:visualization}).

Non-overlapping verification using axis-aligned bounding box collision detection confirmed zero geometric intersections across all scenarios. For the most complex scenario S5 with 300 items, pairwise collision checking validated 44,850 potential item pairs without detecting any overlaps. Boundary containment verification confirmed that all item coordinates remained within container dimensions (12,032 × 2,352 × 2,698 mm), with no items exceeding wall boundaries or floor limits. These geometric validations were performed automatically during the packing computation, with results persisted alongside placement coordinates for subsequent verification.

Stability validation confirmed that all elevated items (items with z-position > 0) achieved at least 75\% base support area from underlying items or the container floor, consistent with the configured \texttt{support\_surface\_ratio} parameter. Visual examination through the 3D visualization interface corroborated these results, with no floating or insufficiently supported items observed in any scenario. The step-by-step playback feature further verified adherence to the Bigger First sorting strategy, consistently showing large items (Euro Pallets and Large Crates) placed in early steps before smaller items filled remaining spaces across all test scenarios.

\subsection{Discussion}

\begin{figure}[!htb]
	\centering
	\includegraphics[width=\textwidth]{figures/detailed_metrics.png}
	\caption{Performance metric summary: (a) volume utilization, (b) fill rate, (c) time scalability, (d) volume and weight utilization correlation}
	\label{fig:detailed}
\end{figure}

Testing results demonstrate that the system successfully processes heterogeneous item configurations with consistent placement success (Figure \ref{fig:detailed}). The 100\% fill rate across all scenarios confirms that the Bigger First heuristic with stability constraints produces reliable packing solutions without placement failures. Deterministic algorithm behavior, evidenced by zero variance in utilization metrics across repeated executions, ensures that generated plans are reproducible and verifiable before physical loading commences. The observed volume-to-weight utilization ratio of approximately 2:1 indicates that the test configuration is volume-constrained, a characteristic common in logistics scenarios involving lightweight manufactured goods.

From an operational perspective, the step-by-step visualization capability bridges the gap between algorithmic computation and physical execution requirements. Operators can follow the chronological placement sequence through the 3D interface without requiring interpretation of raw coordinate data or specialized desktop software installation. The stability checking mechanism, which adds approximately 8\% computational overhead compared to unconstrained packing, provides assurance that generated arrangements are physically feasible with adequate base support. Computation times under one minute for scenarios up to 300 items remain practical for offline planning workflows where plans are prepared in advance of loading operations.

This study focused on demonstrating the feasibility of web-based interactive load planning using deterministic heuristic methods. Performance evaluation relied on automated geometric validation and algorithmic metrics rather than empirical measurement of operator efficiency in field conditions. The deterministic approach prioritizes computational speed and result reproducibility over solution space exploration that metaheuristic methods would provide \cite{Mihu2024}. Detailed discussion of study limitations and directions for future research is presented in the Conclusion.

