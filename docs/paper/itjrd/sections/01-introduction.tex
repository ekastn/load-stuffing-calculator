\section{Introduction}
\label{}
Container loading represents a critical activity in logistics operations that directly influences transportation efficiency and operational costs. Optimal loading maximizes space utilization, reduces trip frequency, and lowers operational expenses and carbon emissions \cite{Tresca2022}. However, loading operations for heterogeneous cargo remain predominantly manual \cite{Ananno2024}. Case studies in chemical manufacturing industries reveal that shipping managers often plan routes and loading arrangements based solely on personal experience due to the absence of decision support tools, resulting in inefficient distance traveled and increased vehicle costs \cite{Jomthong2024}. This manual approach presents several fundamental limitations: it requires specialized training, produces efficiency highly dependent on individual operator capabilities, and cannot guarantee consistent load stability \cite{Ananno2024}.
\par
The container loading problem is formally known as the Three-Dimensional Bin Packing Problem (3D-BPP), an optimization challenge of placing a set of items into containers with the objective of minimizing empty space or the number of required containers. 3D-BPP is classified as NP-hard, meaning computational complexity increases exponentially as problem size grows \cite{Ma2025}. Complexity escalates further when considering practical constraints such as maximum weight limits, load stability, item orientation restrictions, and load-bearing capacity of individual items \cite{Silva2016}. The problem becomes particularly challenging when cargo consists of numerous item types with highly variable dimensions, as commonly encountered in e-commerce logistics and courier service operations \cite{Ma2025}.
\par
Various approaches have been developed to solve 3D-BPP. Classical methods include heuristic algorithms such as First Fit Decreasing and Best Fit, which place items sequentially according to predetermined rules \cite{Silva2016}. Layer-building methods construct horizontal layers then stack them vertically \cite{Tresca2022}. Matheuristic approaches combine linear programming formulations with heuristics to obtain better solutions within reasonable computation time. Tresca et al. \cite{Tresca2022} noted that commercial warehouse management systems still lack effective algorithms to automate optimal pallet configuration.
\par
Recent developments demonstrate the application of metaheuristics and artificial intelligence for 3D-BPP. Khairuddin et al. \cite{Khairuddin2020} developed a Genetic Algorithm (GA)-based simulator that visualizes the optimization process. Ma et al. \cite{Ma2025} combined block-building heuristics, GA, and simulated annealing to handle heterogeneous cargo. Ananno and Ribeiro \cite{Ananno2024} proposed a two-stage algorithm tested with real industrial data. From the artificial intelligence perspective, Deep Reinforcement Learning (DRL) has begun to be applied to 3D-BPP. Zhao et al. \cite{Zhao2022} employed DRL with stacking trees for stability analysis, while Murdivien and Um \cite{Murdivien2023} utilized game engines for training simulation. Hybrid heuristic-DRL approaches have also shown promising results \cite{Wong2024}. Zhang et al. \cite{Zhang2024} proposed a combination of Generative Adversarial Networks with GA to improve solution quality.
\par
Visualization plays an important role in bin packing solutions as it enables result validation and provides guidance to operators. Krebs and Ehmke \cite{Krebs2023} developed an open-source solution validation and visualization tool, however this tool is desktop-based and targeted at researchers rather than operational field guidance. Beyond interface limitations, another significant challenge is algorithm testing that often uses only random datasets that inadequately reflect real-world logistics scenarios \cite{Ribeiro2023}. A gap exists for web-based systems that integrate bin packing algorithms with interactive 3D visualization capable of handling realistic data scenarios and providing step-by-step loading guidance.
\par
This research develops a web-based container loading planning system to bridge this gap. The system utilizes a 3D bin packing library providing stability checking and gravity simulation features, integrating it through a layered architecture. Interactive 3D visualization displays algorithmic calculation results and provides step-by-step loading guidance, enabling operators to visualize the item placement sequence. The system is accessible through web browsers without requiring specialized software installation, thereby improving operational accessibility.
\par
This research has three primary objectives. First, to integrate a 3D bin packing library into a web-based system. Second, to develop interactive 3D visualization with loading sequence playback features for operator guidance. Third, to evaluate system performance through testing with realistic loading scenarios to measure achieved space utilization and computation time.
