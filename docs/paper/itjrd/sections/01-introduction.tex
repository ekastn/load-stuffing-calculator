\section{Introduction}
\label{sec:introduction}
Container loading represents a critical activity in logistics operations that directly influences transportation efficiency and operational costs. The global container shortage intensified since 2021 due to the COVID-19 pandemic, coupled with disruptions such as the 2024 Red Sea shipping crisis that caused freight rate increases up to 173\% for Asia-Europe routes, has further emphasized the importance of optimal space utilization \cite{Hoa2024}. Optimal loading maximizes space utilization, reduces trip frequency, and lowers operational expenses and carbon emissions \cite{Tresca2022}. However, loading operations for heterogeneous items remain predominantly manual \cite{Ananno2024}, with shipping managers often planning arrangements based solely on personal experience due to the absence of decision support tools. This manual approach produces efficiency highly dependent on individual operator capabilities and cannot guarantee consistent load stability—a safety concern particularly critical in logistics operations \cite{Ananno2024}. The container loading problem is formally known as the Three-Dimensional Bin Packing Problem (3D-BPP), an optimization challenge of placing items into containers to minimize empty space or the number of required containers. 3D-BPP is classified as NP-hard, meaning computational complexity increases exponentially as problem size grows \cite{Ma2025}. Complexity escalates when considering practical constraints such as maximum weight limits, load stability requirements, item orientation restrictions, and load-bearing capacity \cite{Krebs2023}. The problem becomes particularly challenging with numerous heterogeneous item types of highly variable dimensions, as commonly encountered in diverse logistics sectors from e-commerce to manufacturing operations \cite{Ma2025, Fontaine2023}. Various approaches have been developed to solve 3D-BPP, including classical heuristic algorithms that place items sequentially according to predetermined rules \cite{Silva2016}, layer-building methods that construct horizontal layers then stack them vertically \cite{Tresca2022}, and metaheuristic approaches using genetic algorithms that respect practical stability constraints \cite{Ananno2024, Hoa2024}. Recent Deep Reinforcement Learning advances have shown improvements in space utilization and computational efficiency \cite{Zhao2022, Wong2024, Zhang2024}.
\par
Despite these algorithmic advances, a critical gap remains in translating optimization solutions into practical operational guidance. Existing validation and visualization tools are desktop-based and primarily targeted at researchers for algorithm verification rather than operational field use \cite{Krebs2023}. While web-based 3D visualization for container loading has been demonstrated \cite{Poerwandono2023}, these implementations lack integrated step-by-step loading guidance that connects algorithmic results with physical execution sequences for warehouse operators. Furthermore, algorithm evaluation often relies on randomly generated datasets that inadequately reflect the heterogeneous item distributions characteristic of real warehouse operations \cite{Ribeiro2023}. A gap exists for web-based systems that integrate bin packing algorithms with interactive 3D visualization, provide step-by-step loading guidance accessible without specialized software installation, and validate performance using realistic heterogeneous scenarios reflective of actual logistics operations.
\par
This research develops a web-based container loading planning system with three primary objectives. First, to design and implement a layered architecture that separates the 3D bin packing computation engine—implementing stability checking and gravity simulation—from data management and presentation layers. Second, to develop interactive 3D visualization using WebGL-based rendering that provides step-by-step loading guidance, enabling operators to visualize the item placement sequence chronologically. Third, to evaluate system effectiveness using realistic heterogeneous item scenarios, measuring achieved space utilization, computation time scalability, and algorithm configuration impact on loading performance. Unlike existing desktop-based tools targeted at researchers \cite{Krebs2023} or standalone visualization implementations \cite{Poerwandono2023}, this system provides an integrated solution accessible through standard web browsers without specialized software installation, addressing the operational gap where algorithmic solutions lack practical guidance for warehouse staff \cite{Tresca2022, Ananno2024}.
