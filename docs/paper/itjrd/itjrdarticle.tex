%%class
\documentclass{itjrdarticle}

%%required package. add for your convenient, but do not remove the initial
\usepackage[none]{hyphenat}
\usepackage{amsmath, amsfonts, amssymb, float, fancyhdr}
\usepackage[figuresright]{rotating}
\usepackage{authblk, graphicx, indentfirst, lastpage, lipsum}
\setlength{\affilsep}{0cm}
\renewcommand\Authfont{\normalsize}
\renewcommand\Affilfont{\normalfont\small}
\usepackage{subfig, caption, epstopdf}
\usepackage[left=3cm, right=2.5cm, top=2.5cm, bottom=2.5cm, includehead, includefoot]{geometry}
\usepackage{caption}
\captionsetup{labelsep=period}
\usepackage{titlesec}
\titleformat{\section}
  {\normalfont\normalsize\bfseries\uppercase}{\thesection}{1em}{}
\titlespacing*{\section}{0cm}{0.7cm}{0cm}

%%additional packages for tables and algorithms
\usepackage{tabularx, booktabs}
% \usepackage{algorithm}
% \usepackage{algpseudocode}
\usepackage{microtype}

\usepackage[ruled,vlined,linesnumbered]{algorithm2e}


%%leave copyright info to the editor
\CopyrightLine[]{}{This work is licensed under a Creative Commons Attribution-ShareAlike 4.0 International License.}

%%author
\author{\bfseries Muhamad Saladin Eka Septian$^1$}
\author{\bfseries Dina Oktafiani$^2$}
\author{\bfseries Roni Andarsyah$^3$}
%%author's affiliation
\affil{Department of Informatics Engineering, Universitas Logistik dan Bisnis Internasional$^{1,2,3}$}
\email{ekaseptian.c@gmail.com$^1$, dinaoktafiani04@gmail.com$^2$, roniandarsyah@ulbi.ac.id$^3$}
%%title and shortitle (for footer)
\title{Interactive 3D Bin Packing System for Optimized Container Loading Operations}  
\shorttitle{Interactive 3D Bin Packing System for Container Loading (Septian)}

%%starting
\begin{document}

%%indentation. do not change
\setlength{\parindent}{1.27cm}

%%header and footer setting. do not change
\pagestyle{fancy}
\fancyhfoffset{0cm}

%%journal info
\journalname{IT Journal Research and Development (ITJRD)}
\journalshortname{IT Jou Res and Dev}
\journalhomepage{http://journal.uir.ac.id/index.php/ITJRD}
\vol{x}
\no{x}
\doi{xxxxx}
\months{Month}
\years{20xx}
\issn{2528-4053}

%%build title
\maketitle

%%border setting. do not change
\hrule
\vspace{.1em}
\hrule
\vspace{.5em}
\noindent
\parbox[t][][s]{0.275\textwidth}{%
	\textbf{Article Info}
	\vspace{.5em}
	\hrule
	\vspace{.5em}
	\begin{history}

		%%article info. editor's privilege
		Received xx, 20xx
		\par
		Revised xx, 20xx
		\par
		Accepted xx, 20xx

	\end{history}
	\vspace{.5em}
	\hrule
	\vspace{.5em}
	\begin{keyword}

		%%write keyword here. separate by \sep
		3D bin packing \sep container loading \sep logistics optimization \sep heuristic algorithm \sep web visualization

		\vspace{.5em}
	\end{keyword}
	\vspace{\fill}
}
\parbox{0.025\textwidth}{\hspace{0.5em}}
\parbox[t][][s]{0.7\textwidth}{%
	\begin{abstract}
		%% Text of abstract
		Manual container loading operations remain inefficient due to reliance on operator experience without decision support tools. This research develops Load \& Stuffing Calculator, a web-based system integrating 3D Bin Packing Problem algorithm with interactive visualization for container loading optimization. The system employs layered architecture separating the packing calculation service from the data management backend, with Three.js-based 3D visualization providing step-by-step loading guidance. Testing across five scenarios (50-300 heterogeneous items) demonstrates consistent 100\% fill rate with volume utilization reaching 55.26\%, while computation time remains under 40 seconds for offline planning requirements. Algorithm comparison confirms Bigger First strategy achieves 42.9\% higher utilization than Smaller First approach. Empirically, this research validates the feasibility of web-based 3D bin packing systems for practical logistics operations. Conceptually, the integration of step playback visualization bridges the gap between algorithmic calculation and physical execution, addressing the limitation of existing tools that lack operational guidance for warehouse staff.
	\end{abstract}
}
\parbox[l]{\textwidth}{%
	\rule{0.275\textwidth}{0.5pt} \hspace{0.5cm} \hrulefill
	\\
	\emph{\textbf{Corresponding Author:}}\\
	%% correspondence info. separate by \\
	Muhamad Saladin Eka Septian\\
	Department of Informatics Engineering\\
	Universitas Logistik dan Bisnis Internasional\\
	Jl. Sariasih No. 54, Sarijadi, Bandung, West Java, Indonesia\\
	ekaseptian.c@gmail.com
}
\vspace{.5em}
\hrule
\vspace{.1em}
\hrule

%% main text

\section{Introduction}
\label{sec:introduction}
Container loading represents a critical activity in logistics operations that directly influences transportation efficiency and operational costs. The global container shortage intensified since 2021 due to the COVID-19 pandemic, coupled with disruptions such as the 2024 Red Sea shipping crisis that caused freight rate increases up to 173\% for Asia-Europe routes, has further emphasized the importance of optimal space utilization \cite{Hoa2024}. Optimal loading maximizes space utilization, reduces trip frequency, and lowers operational expenses and carbon emissions \cite{Tresca2022}. However, loading operations for heterogeneous items remain predominantly manual \cite{Ananno2024}, with shipping managers often planning arrangements based solely on personal experience due to the absence of decision support tools. This manual approach produces efficiency highly dependent on individual operator capabilities and cannot guarantee consistent load stability—a safety concern particularly critical in logistics operations \cite{Ananno2024}. The container loading problem is formally known as the Three-Dimensional Bin Packing Problem (3D-BPP), an optimization challenge of placing items into containers to minimize empty space or the number of required containers. 3D-BPP is classified as NP-hard, meaning computational complexity increases exponentially as problem size grows \cite{Ma2025}. Complexity escalates when considering practical constraints such as maximum weight limits, load stability requirements, item orientation restrictions, and load-bearing capacity \cite{Krebs2023}. The problem becomes particularly challenging with numerous heterogeneous item types of highly variable dimensions, as commonly encountered in diverse logistics sectors from e-commerce to manufacturing operations \cite{Ma2025, Fontaine2023}. Various approaches have been developed to solve 3D-BPP, including classical heuristic algorithms that place items sequentially according to predetermined rules \cite{Silva2016}, layer-building methods that construct horizontal layers then stack them vertically \cite{Tresca2022}, and metaheuristic approaches using genetic algorithms that respect practical stability constraints \cite{Ananno2024, Hoa2024}. Recent Deep Reinforcement Learning advances have shown improvements in space utilization and computational efficiency \cite{Zhao2022, Wong2024, Zhang2024}.
\par
Despite these algorithmic advances, a critical gap remains in translating optimization solutions into practical operational guidance. Existing validation and visualization tools are desktop-based and primarily targeted at researchers for algorithm verification rather than operational field use \cite{Krebs2023}. While web-based 3D visualization for container loading has been demonstrated \cite{Poerwandono2023}, these implementations lack integrated step-by-step loading guidance that connects algorithmic results with physical execution sequences for warehouse operators. Furthermore, algorithm evaluation often relies on randomly generated datasets that inadequately reflect the heterogeneous item distributions characteristic of real warehouse operations \cite{Ribeiro2023}. A gap exists for web-based systems that integrate bin packing algorithms with interactive 3D visualization, provide step-by-step loading guidance accessible without specialized software installation, and validate performance using realistic heterogeneous scenarios reflective of actual logistics operations.
\par
This research develops a web-based container loading planning system with three primary objectives. First, to design and implement a layered architecture that separates the 3D bin packing computation engine—implementing stability checking and gravity simulation—from data management and presentation layers. Second, to develop interactive 3D visualization using WebGL-based rendering that provides step-by-step loading guidance, enabling operators to visualize the item placement sequence chronologically. Third, to evaluate system effectiveness using realistic heterogeneous item scenarios, measuring achieved space utilization, computation time scalability, and algorithm configuration impact on loading performance. Unlike existing desktop-based tools targeted at researchers \cite{Krebs2023} or standalone visualization implementations \cite{Poerwandono2023}, this system provides an integrated solution accessible through standard web browsers without specialized software installation, addressing the operational gap where algorithmic solutions lack practical guidance for warehouse staff \cite{Tresca2022, Ananno2024}.

\section{Research Method}
\label{}
This research methodology encompasses mathematical model formulation, system architecture design, algorithm integration procedure implementation, and system testing scenario design.

\subsection{Theoretical Foundation and 3D-BPP Modeling}
The Three-Dimensional Bin Packing Problem (3D-BPP) represents a combinatorial optimization problem focused on placing a set of three-dimensional items into containers (bins) to maximize available space utilization \cite{Ma2025}. This problem is NP-hard, meaning solution search complexity increases exponentially as the number and dimensional variation of items grows \cite{Silva2016}. Therefore, this research applies a heuristic approach to obtain efficient solutions within real logistics operational time constraints.

In this system's mathematical modeling, a container is defined with dimensions length ($L$), width ($W$), and height ($H$) with maximum load capacity $M$. For each item $i$ from a total of $n$ items, there are dimensional attributes $(l_i, w_i, h_i)$, volume $v_i$, and mass $m_i$. Decision variables include $\eta_i \in \{0,1\}$ as an indicator of whether item $i$ is successfully loaded, position coordinates $(x_i, y_i, z_i)$, and orientation indicator $r_{i,p}$ to handle possible item rotations.

The primary objective of this system is expressed in the objective function in Equation \ref{eq:objective}:
\begin{equation}
	\max \sum_{i=1}^{n} v_i \cdot \eta_i
	\label{eq:objective}
\end{equation}

Equation \ref{eq:objective} explains that the algorithm's main target is to maximize the total volume of all items successfully placed ($\eta_i = 1$) within a single container. By maximizing the total volume of loaded items, the system automatically minimizes unused empty space (void space), which in turn reduces transportation cost per unit of goods.

To measure the effectiveness of these calculation results, a space utilization metric ($U$) is used, formulated in Equation \ref{eq:utilization}:
\begin{equation}
	U = \frac{\sum_{i=1}^{n} v_i \cdot \eta_i}{L \times W \times H} \times 100\%
	\label{eq:utilization}
\end{equation}

Equation \ref{eq:utilization} represents the loading efficiency ratio in percentage form. The numerator in this formula is the total volume of successfully loaded items, while the denominator is the total container capacity volume ($L \times W \times H$). A value of $U$ approaching 100\% indicates that the system successfully arranges goods very densely and efficiently.

To ensure that generated solutions are not only volume-optimal but also physically stable, this model integrates a number of operational constraints. Table \ref{tab:constraints} summarizes these constraints, where stability constraint receives special attention by requiring a minimum support area ($\theta$) of 75\% of the item's base area to prevent cargo shifting during distribution \cite{Ananno2024}.

\begin{table}[H]
	\centering
	\caption{Operational Constraints in 3D-BPP}
	\label{tab:constraints}
	\small
	\begin{tabularx}{\textwidth}{lX}
		\toprule
		\textbf{Constraint} & \textbf{Formulation and Description}                                                                                                         \\
		\midrule
		Volume              & $\sum_{i=1}^{n} v_i \cdot \eta_i \leq L \times W \times H$. Total item volume does not exceed container capacity.                            \\
		Mass                & $\sum_{i=1}^{n} m_i \cdot \eta_i \leq M$. Total weight does not exceed maximum load capacity.                                                \\
		Orientation         & $\sum_{p=1}^{6} r_{i,p} = 1, \; \forall i$. Each item selects exactly one of six rotation orientations.                                      \\
		Non-overlap         & $(x_i + d_{ix} \leq x_j) \lor (x_j + d_{jx} \leq x_i) \lor \dots, \; \forall i \neq j$. Items must not occupy the same space.                \\
		Stability           & $z_c = 0 \;\lor\; \sum_{c' \in S_z} \text{OverlapArea}(c, c') \geq \theta \cdot A_c$. Items must be supported by at least 75\% of base area. \\
		Load Bearing        & $D_i \leq R_i$, where $D_i = \sum_{j \in C_i}(m_j + D_j)$. Cumulative load above item does not exceed load-bearing capacity.                 \\
		Center of Gravity   & $\text{con}_{x1} \leq G_x \leq \text{con}_{x2}$, etc. Center of gravity of cargo remains within safe zone (optional).                        \\
		\bottomrule
	\end{tabularx}
\end{table}


\subsection{System Architecture and Design}

The Load \& Stuffing Calculator system is built using layered architecture to separate operational data management from heavy computation engines. This architecture is designed to ensure system scalability when handling complex cargo calculations. The system is divided into four components as illustrated in Figure \ref{fig:architecture}: (1) Web Frontend, (2) Backend API, (3) Database, and (4) Packing Service.

\begin{figure}[H]
	\centering
	\includegraphics[width=\textwidth]{figures/architecture0.pdf}
	\caption{Container loading planning system architecture}
	\label{fig:architecture}
\end{figure}

The first component is the \textbf{Web Frontend} which handles user interaction and loading result visualization. There are three user roles: Admin for user management and access rights, Planner for creating loading plans, and Operator for viewing loading guidance. An interactive 3D visualization module was developed for WebGL-based rendering. Unlike conventional visualization tools that are often static and aimed solely at researcher validation \cite{Krebs2023}, this system is specifically designed to assist field operators in executing physical loading plans through a step playback feature.

The second component is the \textbf{Backend API} which acts as the system's main gateway. This component handles user authentication through JWT (JSON Web Token) and authorization through Role-Based Access Control (RBAC). The handler layer receives HTTP requests and forwards them to the service layer for business logic processing. The data access layer then interacts with the database or calls external services through the Packing Gateway.

The third component is the \textbf{Database} which stores all persistent system data. Data is grouped into three categories: (1) authentication and RBAC data (\texttt{users}, \texttt{roles}, \texttt{permissions}); (2) master data (\texttt{products}, \texttt{containers}); and (3) planning data (\texttt{load\_plans}, \texttt{load\_items}, \texttt{plan\_placements}).

The fourth component is the \textbf{Packing Service} which acts as a separate algorithmic calculation engine. This separation is performed so that business logic in the main API is not hindered by intensive 3D-BPP computation processes. This service encapsulates the 3D bin packing library and communicates with the backend through REST HTTP protocol. Calculation results in the form of a placement list with position coordinates and step sequence numbers are returned to the backend for storage and display in visualization (Figure \ref{fig:visualization}).

\begin{figure}[h]
	\centering
	\includegraphics[width=\textwidth]{figures/antarmuka.png}
	\caption{3D visualization interface with playback controls and loading summary}
	\label{fig:visualization}
\end{figure}

This visualization design includes geometric representation of items as BoxGeometry objects with color codes representing different product types. A crucial implemented feature is the loading sequence playback mechanism that utilizes the \texttt{step\_number} variable from calculation results. This mechanism enables the system to display items one by one according to the optimal placement sequence generated by the algorithm, allowing operators to follow step-by-step loading guidance intuitively. This addresses the operational gap where 3D visualization often does not provide practical loading sequence context for warehouse staff \cite{Krebs2023}.

\subsection{Algorithm and Loading Integration}

This section explains the core mechanism of the packing service that integrates heuristic algorithms into the system workflow. The system implements an algorithm based on block-building heuristic modified to handle real-world physical constraints \cite{Silva2016, Ma2025}. The primary strategy employed is Bigger First, where items with the largest volumes are prioritized for placement first to ensure efficient space utilization from the beginning of the loading process.

Algorithm integration is performed through a Python service that encapsulates functions from the 3D bin packing library. There are three key features activated in each calculation: (1) \textbf{Fix Point} which simulates gravity to ensure items are always at the lowest valid position; (2) \textbf{Check Stable} to validate the support area of items beneath; and (3) \textbf{Rotation Optimization} which evaluates six possible item orientations to find the best space fit. The data transformation flow from request to final result is illustrated in Figure \ref{fig:flowchart}.

\begin{figure}[h]
	\centering
	\includegraphics[width=\textwidth]{figures/flowchart.pdf}
	\caption{Loading transformation and calculation process flow}
	\label{fig:flowchart}
\end{figure}

One technical challenge in this integration is the difference in coordinate axis conventions between system components. The calculation library uses the $Y$ axis as height representation, while this system's API uses the $Z$ axis as height according to logistics coordinate standards. Furthermore, the Three.js frontend uses the $Y$-up convention ($Y$ axis as vertical), necessitating dual transformation. In the packing service, internal coordinates $(p_x, p_y, p_z)$ are mapped to API coordinates as $x = p_x$, $y = p_z$, $z = p_y$. Then in the frontend, API coordinates $(x, y, z)$ are transformed to Three.js as $X_{3D} = x$, $Y_{3D} = z$, $Z_{3D} = -y$. This integration procedure is summarized in Algorithm \ref{alg:integration}.

% \begin{algorithm}[H]
% 	\caption{Bin Packing Algorithm Integration Procedure}
% 	\label{alg:integration}
% 	\begin{algorithmic}[1]
% 		\Require container dimensions, item list, option parameters (support\_ratio)
% 		\Ensure sorted placement list
% 		\State Convert all item and container dimension units to centimeters (cm)
% 		\State Initialize Packer object and add container (Bin)
% 		\For{each item in request list}
% 		\State Add item to Packer queue based on its quantity
% 		\EndFor
% 		\State Execute \Call{Packer.pack}{bigger\_first, fix\_point, check\_stable}
% 		\For{each successfully loaded item (based on putOrder sequence)}
% 		\State Perform axis transformation: $x=p_x, y=p_z, z=p_y$
% 		\State Calculate rotation\_code and assign step\_number
% 		\State Save data to placement result collection
% 		\EndFor
% 		\State \Return placements in JSON format
% 	\end{algorithmic}
% \end{algorithm}

\begin{algorithm}[!htb]
	\caption{Bin Packing Algorithm Integration Procedure}
	\label{alg:integration}
	\KwIn{container dimensions, item list, option parameters (support\_ratio)}
	\KwOut{sorted placement list}
	
	Convert all item and container dimension units to centimeters (cm)\;
	Initialize Packer object and add container (Bin)\;
	
	\ForEach{item in request list}{
		Add item to Packer queue based on its quantity\;
	}
	
	Execute Packer.pack(bigger\_first, fix\_point, check\_stable)\;
	
	\ForEach{successfully loaded item (based on putOrder sequence)}{
		Perform axis transformation: $x=p_x, y=p_z, z=p_y$\;
		Calculate rotation\_code and assign step\_number\;
		Save data to placement result collection\;
	}
	
	\Return placements in JSON format\;
\end{algorithm}

\subsection{Evaluation Method and Testing}

System evaluation was performed to measure system performance in handling various loading complexity levels. Testing was designed using realistic scenarios according to literature recommendations for logistics algorithm testing \cite{Ribeiro2023}. Test scenarios (S1--S5) were structured based on increasing item counts and product heterogeneity levels (Table \ref{tab:scenarios}). This aims to test computation response time scalability and algorithm consistency in maintaining optimal space utilization \cite{Jomthong2024}.

\begin{table}[h]
	\centering
	\caption{System Performance Test Scenarios}
	\label{tab:scenarios}
	\begin{tabular}{clll}
		\toprule
		\textbf{Scenario} & \textbf{Items} & \textbf{Heterogeneity} & \textbf{Evaluation Purpose}                   \\
		\midrule
		S1                & 50             & Homogeneous            & Functional validation and coordinate accuracy \\
		S2                & 100            & Light Heterogeneous    & Standard operational case testing             \\
		S3                & 150            & Moderate Heterogeneous & Medium-level computational load testing       \\
		S4                & 200            & Highly Heterogeneous   & High complexity testing                       \\
		S5                & 300            & Highly Heterogeneous   & System scalability limit testing              \\
		\bottomrule
	\end{tabular}
\end{table}

Primary evaluation metrics include: (1) \textbf{Space Utilization} ($U$) calculated through Equation \ref{eq:utilization}; (2) \textbf{Computation Time} in milliseconds ($ms$); and (3) \textbf{Fill Rate} which measures the ratio of successfully loaded items compared to total requests. Testing also includes visual validation on the 3D interface to ensure absence of anomalies such as overlapping items or floating items due to gravity constraint violations.

\section{Results and Discussion}
\label{sec:results}

Testing was conducted based on the scenarios defined in the methodology. Results address space utilization, computation time scalability, algorithm variant performance, and visual validation.

\subsection{Experimental Setup}

Testing was conducted on a system with Intel Core i5-6440HQ @ 2.60GHz processor, 16 GB RAM, and Linux operating system. The application used Python 3.11 for the packing service and Go 1.21 for the backend API. Each scenario was executed five times to obtain mean values and standard deviations.

The container used throughout testing was a 40-foot High Cube type with internal dimensions of 12.032 $\times$ 2.352 $\times$ 2.698 meters (volume 76.35 m\textsuperscript{3}) and maximum load capacity of 26,460 kg. Four product types with varying dimensions were defined to simulate item heterogeneity: Euro Pallet (1200$\times$800$\times$144 mm), Large Crate (1000$\times$600$\times$500 mm), Medium Box (600$\times$400$\times$400 mm), and Small Box (400$\times$300$\times$200 mm). Each scenario uses a subset of these product types as specified in Table \ref{tab:scenarios}.

\subsection{Performance Results}

Table \ref{tab:results} presents testing results for all five scenarios using the baseline algorithm configuration (Bigger First strategy with stability checking enabled). Measured metrics include volume utilization, weight utilization, fill rate, and computation time.

\begin{table}[h]
	\centering
	\caption{Loading Algorithm Performance Testing Results}
	\label{tab:results}
	\small
	\begin{tabular}{lrrrrr}
		\toprule
		\textbf{Scenario} & \textbf{Items} & \textbf{Vol. Util. (\%)} & \textbf{Weight Util. (\%)} & \textbf{Fill Rate (\%)} & \textbf{Time (ms)} \\
		\midrule
		S1                & 50             & 6.29 $\pm$ 0.00          & 2.83 $\pm$ 0.00            & 100.00                  & 320 $\pm$ 6        \\
		S2                & 100            & 8.80 $\pm$ 0.00          & 4.16 $\pm$ 0.00            & 100.00                  & 1,684 $\pm$ 131    \\
		S3                & 150            & 24.83 $\pm$ 0.00         & 11.15 $\pm$ 0.00           & 100.00                  & 5,213 $\pm$ 169    \\
		S4                & 200            & 35.45 $\pm$ 0.00         & 16.25 $\pm$ 0.00           & 100.00                  & 10,826 $\pm$ 168   \\
		S5                & 300            & 55.26 $\pm$ 0.00         & 25.32 $\pm$ 0.00           & 100.00                  & 38,343 $\pm$ 1,209 \\
		\bottomrule
	\end{tabular}
\end{table}

The standard deviation of 0.00 for utilization metrics reflects the deterministic nature of the algorithm, where identical inputs produce identical placement results. Only computation time exhibits variance due to system-level factors such as process scheduling and memory allocation.

The algorithm successfully placed all items (100\% fill rate) across all scenarios, indicating that the 40-foot High Cube container capacity is adequate for the tested item configurations. The relatively low utilization values in scenarios S1 and S2 reflect the test design rather than algorithm inefficiency; these scenarios intentionally use fewer items than required to fill the container, allowing assessment of algorithm behavior across varying load levels. Volume utilization increases proportionally with item count, from 6.29\% in S1 to 55.26\% in S5. Weight utilization values consistently lower than volume utilization indicate that the configuration is volume-constrained rather than weight-constrained.

Figure \ref{fig:utilization} visualizes the comparison of volume and weight utilization for each scenario. Volume utilization consistently exceeds weight utilization with approximately a 2:1 ratio, confirming the volume-constrained characteristics of the test configuration.

\begin{figure}[H]
	\centering
	\includegraphics[width=0.9\textwidth]{figures/utilization_comparison.pdf}
	\caption{Comparison of volume and weight utilization in each scenario}
	\label{fig:utilization}
\end{figure}

\subsection{Computation Time Analysis}

An important consideration in 3D-BPP evaluation is computation time scalability with increasing item counts. Table \ref{tab:scalability} presents detailed computation time statistics, including minimum, average, and maximum values for each scenario.

\begin{table}[h]
	\centering
	\caption{Computation Time Scalability Analysis}
	\label{tab:scalability}
	\small
	\begin{tabular}{lrrrr}
		\toprule
		\textbf{Scenario} & \textbf{Items} & \textbf{Min (ms)} & \textbf{Avg (ms)} & \textbf{Max (ms)} \\
		\midrule
		S1                & 50             & 311               & 320               & 325               \\
		S2                & 100            & 1,489             & 1,684             & 1,791             \\
		S3                & 150            & 5,012             & 5,213             & 5,391             \\
		S4                & 200            & 10,544            & 10,826            & 10,972            \\
		S5                & 300            & 36,334            & 38,343            & 39,288            \\
		\bottomrule
	\end{tabular}
\end{table}

Figure \ref{fig:scalability} illustrates the relationship between item count and computation time. Computation time exhibits quadratic growth ($O(n^2)$), consistent with heuristic-based bin packing algorithm characteristics \cite{Ma2025}. For each item placement, the algorithm evaluates potential collisions with all previously positioned items, resulting in quadratic complexity.

\begin{figure}[H]
	\centering
	\includegraphics[width=0.9\textwidth]{figures/computation_time.pdf}
	\caption{Computation time scalability with respect to item count}
	\label{fig:scalability}
\end{figure}

Although computation time in scenario S5 (300 items) reaches an average of 38 seconds, this value remains acceptable for offline loading planning applications. In logistics operational contexts, loading planning is typically performed before physical loading commences, making wait times under one minute acceptable for operational requirements \cite{Tresca2022}.

\subsection{Algorithm Variant Comparison}

To assess the contribution of each algorithm component, comparative testing was performed with three configuration variants on scenario S3 (150 items). Table \ref{tab:variants} presents comparison results between the baseline configuration (Bigger First with Stability), a variant without stability checking, and a variant using the Smaller First strategy.

\begin{table}[h]
	\centering
	\caption{Algorithm Variant Comparison in Scenario S3 (150 items)}
	\label{tab:variants}
	\small
	\begin{tabular}{lrrrr}
		\toprule
		\textbf{Variant}            & \textbf{Vol. Util. (\%)} & \textbf{Fill Rate (\%)} & \textbf{Time (ms)} & \textbf{Items Packed} \\
		\midrule
		Bigger First + Stability    & 24.83                    & 100.00                  & 5,386              & 150                   \\
		Bigger First (No Stability) & 24.83                    & 100.00                  & 4,960              & 150                   \\
		Smaller First + Stability   & 17.37                    & 87.33                   & 32,886             & 131                   \\
		\bottomrule
	\end{tabular}
\end{table}

The comparison reveals that the Bigger First strategy substantially outperforms Smaller First, yielding 43\% relatively higher volume utilization (24.83\% vs. 17.37\%) and achieving complete placement compared to only 87.33\% fill rate. This confirms the heuristic principle that placing large items first leaves small gaps fillable by smaller items, whereas the reverse approach creates suboptimal fragmentation \cite{Ma2025}. The Smaller First variant also requires 6.1 times longer execution (32,886 ms vs. 5,386 ms) due to increased failed placement attempts when small items occupy spaces more suitable for large items.

Disabling stability checking saves only approximately 8\% computation time (4,960 ms vs. 5,386 ms) without affecting utilization results. This indicates that stability checking overhead is relatively small and worth retaining to ensure physical loading safety.

\subsection{Validation Results}

Visual examination through the implemented 3D visualization interface (Figure \ref{fig:visualization}) confirmed that all placements satisfy three validation criteria. No overlapping items were detected in any scenario; position coordinates and dimensions of each item showed no geometric intersections. All items remained within container boundaries without exceeding walls or floor limits. Stability requirements were met, with all elevated items having at least 75\% base area support, consistent with configured \texttt{support\_surface\_ratio} parameters.

The step playback feature enables verification of item placement sequence. Adjusting the step control from 1 to the total item count reveals that large items (Euro Pallet and Large Crate) are placed first, followed by smaller items filling remaining spaces. This pattern is consistent with the configured Bigger First strategy.

\subsection{Comparative Analysis}

\begin{figure}[!htb]
	\centering
	\includegraphics[width=\textwidth]{figures/detailed_metrics.png}
	\caption{Performance metric summary: (a) volume utilization, (b) fill rate, (c) time scalability, (d) volume and weight utilization correlation}
	\label{fig:detailed}
\end{figure}

The developed system successfully processes up to 300 heterogeneous items with 100\% fill rate, achieves volume utilization up to 55.26\%, and completes calculations within 38 seconds for the largest scenario (Figure \ref{fig:detailed}). The system produces physically stable arrangements with minimum 75\% support and provides step-by-step visual guidance to assist operators in physical loading execution.

Comparison with existing literature indicates that the achieved volume utilization (55.26\% in the densest scenario) falls within expected ranges for heterogeneous cases with stability constraints. Table \ref{tab:literature} summarizes performance metrics from related studies for contextual comparison.

\begin{table}[h]
	\centering
	\caption{Comparison with Related Literature}
	\label{tab:literature}
	\small
	\begin{tabularx}{\textwidth}{lXll}
		\toprule
		\textbf{Study}                         & \textbf{Method}          & \textbf{Utilization/Fill Rate} & \textbf{Computation Time} \\
		\midrule
		Ananno \& Ribeiro \cite{Ananno2024}    & Multi-heuristic + GA     & 50--70\%                       & Varies by order           \\
		Ma et al. \cite{Ma2025}                & Block-building + GA + SA & 60--75\%                       & Not specified             \\
		Hoa et al. \cite{Hoa2024}              & GA + Wall-building       & Up to 91.67\% fill             & Not specified             \\
		Tresca et al. \cite{Tresca2022}        & MILP + Layer-building    & High quality                   & $<$30s per bin            \\
		Fontaine \& Minner \cite{Fontaine2023} & Branch-and-repair MILP   & Optimal for bin selection      & 30\% faster than baseline \\
		Zhao et al. \cite{Zhao2022}            & DRL + Stacking tree      & +10\% vs baseline              & Training required         \\
		\textbf{This research}                 & Heuristic (Bigger First) & 55.26\%                        & 38s (300 items)           \\
		\bottomrule
	\end{tabularx}
\end{table}

Ananno and Ribeiro \cite{Ananno2024} reported 50--70\% utilization for industrial cases with similar stability constraints (75\% minimum support area), while Ma et al. \cite{Ma2025} achieved 60--75\% on datasets with more homogeneous items. Hoa et al. \cite{Hoa2024} obtained higher fill rates (up to 91.67\%) for textile industry applications using genetic algorithms with wall-building heuristics, though their configuration involved different constraint priorities including purchase order sequencing; earlier work using simulated annealing achieved approximately 85\% fill rates with deadline-based prioritization \cite{Hoa2024}. The utilization difference in this research can be attributed to the high degree of dimensional heterogeneity in the test configuration, which presents greater packing challenges than homogeneous item sets. Additionally, metaheuristic approaches such as GA typically achieve higher utilization by exploring larger solution spaces at the cost of longer computation times. Direct comparison across studies is inherently limited due to variations in container types, item characteristics, and constraint specifications.

From a computational perspective, the quadratic time complexity observed aligns with findings by Tresca et al. \cite{Tresca2022}, who reported similar scaling behavior for greedy heuristics in container loading problems. Their matheuristic approach achieved configurations in under 30 seconds per bin on average, comparable to the performance observed in this research. Fontaine and Minner \cite{Fontaine2023} demonstrated that MILP-based approaches for e-commerce bin selection can achieve 30\% runtime reduction through branch-and-repair methods, though such exact methods are typically applied to smaller problem instances. The sub-minute computation times for 300 items compare favorably with metaheuristic approaches such as those by Zhao et al. \cite{Zhao2022}, which may require longer execution times for equivalent problem sizes but can yield marginally higher utilization through more exhaustive search. This trade-off between computation speed and solution quality supports the selection of heuristic methods for interactive planning applications where rapid feedback is prioritized over marginal utilization gains.


\section{Conclusion}

This research achieved its three primary objectives in developing a web-based container loading planning system. First, a layered system architecture was successfully designed and implemented, effectively separating data management from the 3D bin packing calculation engine while enabling seamless integration through a RESTful API. The architecture supports role-based access control for administrators, planners, and operators with distinct permission levels. Second, interactive 3D visualization with step-by-step playback was developed using WebGL-based rendering, providing actionable loading guidance that bridges algorithmic optimization with physical execution requirements. The visualization enables operators to follow item placement sequences chronologically without requiring specialized desktop software installation. Third, system effectiveness was validated through five heterogeneous item scenarios with varying complexity levels, demonstrating practical applicability for offline warehouse operational planning.

Experimental evaluation revealed several key quantitative findings. The system achieved 55.26\% volume utilization in the most complex scenario featuring four distinct product types and 300 item units, with placement success rates (fill rate) consistently reaching 100\% across all test scenarios. The Bigger First sorting strategy combined with static stability constraints produced optimal space filling, outperforming Smaller First by 43\% in volume utilization. This performance aligns with literature findings where similar deterministic approaches achieve 60--85\% utilization depending on item heterogeneity. Stability checking added minimal computational overhead of approximately 8\% while ensuring physical feasibility through 75\% minimum base area support requirements. Computation time exhibited quadratic growth patterns with item population, reaching approximately 38 seconds for 300 units---a duration well within tolerance limits for offline planning operations where plans are prepared before physical loading begins.

Several limitations should be acknowledged. The current implementation employs a deterministic heuristic algorithm without solution space exploration capabilities that metaheuristic approaches such as genetic algorithms or simulated annealing would provide, potentially limiting achievable utilization rates. Test scenarios utilized standard rectangular items, excluding irregular or deformable shapes \cite{Zuo2022} that occur in certain logistics contexts such as fresh food delivery. Furthermore, while the system was tested with realistic heterogeneous item configurations, no formal user study with actual warehouse operators was conducted to validate usability and practical effectiveness in field conditions. Weight constraints, although implemented, were not tested at capacity limits across scenarios.

Future research can extend this work in multiple directions. Integration with metaheuristic optimization algorithms such as genetic algorithms or simulated annealing \cite{Tsao2024} could improve space utilization beyond deterministic heuristic limits through broader solution space exploration. Deep Reinforcement Learning approaches offer potential for adaptive online packing decisions that respond to real-time item arrivals, as demonstrated by recent advances \cite{Zhao2022, Murdivien2023} achieving up to 10\% utilization improvements over conventional methods. IoT sensor integration would enable automatic item dimension capture and real-time tracking during loading operations, facilitating validation between planned and actual placements. A formal user study with warehouse operators would provide empirical validation of the visualization interface effectiveness and identify areas for usability improvement. Finally, mobile interface development would enhance field accessibility for operators who require step-by-step guidance during physical loading execution.


\section*{Acknowledgement}
\label{}
The acknowledgment section is optional. The funding source of the research can be put here.

%% The Appendices part is started with the command \appendix;
%% appendix sections are then done as normal sections
%% \appendix

%% \section{}
%% \label{}

%% References
%%
%% Following citation commands can be used in the body text:
%% Usage of \cite is as follows:
%%   \cite{key}         ==>>  [#]
%%   \cite[chap. 2]{key} ==>> [#, chap. 2]
%%

%% References with BibTeX database:

\bibliographystyle{IEEEtran}
\bibliography{references}

%% Authors are advised to use a BibTeX database file for their reference list.
%% The provided style IEEEtran.bst formats references is generally used.

%\section*{BIOGRAPHY OF AUTHORS}

%\begin{biography}[{\includegraphics[width=3cm,height=4cm,clip,keepaspectratio]{First_Author's_Photo}}]
%	\textbf{First Author}
%	%% Affiliation and educational background.
%	obtained Bachelor Degree in Computer Science from University of Somewhere in 2010, obtained Master Degree in Management Information System from University of Somewhere in 2012, and obtained Doctoral of Informatics from University of Somewhere in 2018.
%	He has been a Lecturer with the Department of Informatics Engineering, University of Somewhere, since 2013.
%	%% Research interests, teaching, professional experiences, etc.
%	His current research interests include computational linguistics, natural language processing and machine learning.
%% Related activities and services.
%\end{biography}

%\begin{biography}[{\includegraphics[width=3cm,height=4cm,clip,keepaspectratio]{Second_Author's_Photo}}]
%	\textbf{Second Author}
%% Affiliation and educational background.
%	obtained Bachelor Degree in Computer Science from University of Somewhere in 2010, obtained Master Degree in Management Information System from University of Somewhere in 2012, and obtained Doctoral of Informatics from University of Somewhere in 2018.
%	He has been a Lecturer with the Department of Informatics Engineering, University of Somewhere, since 2013.
%% Research interests, teaching, professional experiences, etc.
%	His current research interests include computational linguistics, natural language processing and machine learning.
%% Related activities and services.
%\end{biography}
\end{document}

%%
%% End of file `iaesarticle2.tex'. 
