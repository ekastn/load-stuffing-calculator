% ============================================================================
% ABSTRACT (English & Indonesian) + Keywords
% ============================================================================

% === ABSTRACT (ENGLISH) ===
\begin{abstract}
	\noindent
	\textit{Manual container loading operations remain inefficient due to reliance on operator experience without decision support tools. This research develops Load \& Stuffing Calculator, a web-based system integrating 3D Bin Packing Problem algorithm with interactive visualization for container loading optimization. The system employs layered architecture separating the packing calculation service from the data management backend, with Three.js-based 3D visualization providing step-by-step loading guidance. Testing across five scenarios (50-300 heterogeneous items) demonstrates consistent 100\% fill rate with volume utilization reaching 55.26\%, while computation time remains under 40 seconds for offline planning requirements. Algorithm comparison confirms Bigger First strategy achieves 42.9\% higher utilization than Smaller First approach. Empirically, this research validates the feasibility of web-based 3D bin packing systems for practical logistics operations. Conceptually, the integration of step playback visualization bridges the gap between algorithmic calculation and physical execution, addressing the limitation of existing tools that lack operational guidance for warehouse staff.}

	\vspace{0.5em}
	\noindent
	\textbf{Keywords:} \textit{3D bin packing, container loading, logistics optimization, heuristic algorithm, web visualization}
\end{abstract}

\vspace{1em}

% === ABSTRAK (INDONESIAN) ===
\noindent
\textbf{Abstrak.}
\textit{Operasi pemuatan kontainer secara manual masih tidak efisien karena bergantung pada pengalaman operator tanpa alat bantu pengambilan keputusan. Penelitian ini mengembangkan Load \& Stuffing Calculator, sistem berbasis web yang mengintegrasikan algoritma 3D Bin Packing Problem dengan visualisasi interaktif untuk optimasi pemuatan kontainer. Sistem menggunakan arsitektur berlapis yang memisahkan layanan kalkulasi dari backend manajemen data, dengan visualisasi 3D berbasis Three.js yang menyediakan panduan pemuatan langkah demi langkah. Pengujian pada lima skenario (50-300 item heterogen) menunjukkan fill rate 100\% secara konsisten dengan utilisasi volume mencapai 55,26\%, sementara waktu komputasi tetap di bawah 40 detik untuk kebutuhan perencanaan offline. Perbandingan algoritma mengkonfirmasi strategi Bigger First mencapai utilisasi 42,9\% lebih tinggi dibandingkan pendekatan Smaller First. Secara empiris, penelitian ini memvalidasi kelayakan sistem 3D bin packing berbasis web untuk operasi logistik praktis. Secara konseptual, integrasi visualisasi step playback menjembatani kesenjangan antara kalkulasi algoritmik dan eksekusi fisik, mengatasi keterbatasan alat yang ada yang kurang menyediakan panduan operasional bagi staf gudang.}

\vspace{0.5em}
\noindent
\textbf{Kata Kunci:} \textit{3D bin packing, pemuatan kontainer, optimasi logistik, algoritma heuristik, visualisasi web}

\vspace{1.5em}

