\section{PENDAHULUAN}

Pemuatan barang ke dalam kontainer merupakan salah satu aktivitas kritis dalam proses logistik yang secara langsung mempengaruhi efisiensi dan biaya transportasi. Pemuatan yang optimal dapat memaksimalkan utilisasi ruang, mengurangi jumlah perjalanan, serta menurunkan biaya operasional dan emisi karbon \cite{Tresca2022}. Namun, operasi pemuatan untuk kargo heterogen saat ini masih didominasi oleh tenaga manusia \cite{Ananno2024}. Studi kasus pada industri manufaktur kimia menunjukkan bahwa manajer pengiriman seringkali merencanakan rute dan pemuatan hanya berdasarkan pengalaman pribadi karena tidak tersedianya alat bantu pengambilan keputusan, yang berakibat pada inefisiensi jarak tempuh serta peningkatan biaya upah kendaraan \cite{Jomthong2024}. Pendekatan manual ini memiliki beberapa kelemahan utama: memerlukan pelatihan khusus, menghasilkan efisiensi yang sangat bergantung pada kemampuan individu operator, serta tidak dapat memberikan jaminan stabilitas susunan barang yang konsisten \cite{Ananno2024}.

Permasalahan pemuatan kontainer secara formal dikenal sebagai \textit{Three-Dimensional Bin Packing Problem} (3D-BPP), yaitu permasalahan optimasi untuk menempatkan sekumpulan item ke dalam kontainer dengan tujuan meminimalkan ruang kosong atau jumlah kontainer yang dibutuhkan. 3D-BPP terklasifikasi sebagai NP-hard sehingga kompleksitas komputasi meningkat secara eksponensial seiring bertambahnya ukuran masalah \cite{Ma2025}. Kompleksitas semakin bertambah ketika mempertimbangkan batasan praktis seperti berat maksimum, stabilitas susunan, orientasi item, serta daya dukung setiap item \cite{Silva2016}. Permasalahan menjadi lebih menantang ketika kargo terdiri dari banyak jenis item dengan dimensi yang sangat bervariasi, sebagaimana lazim ditemui dalam operasi logistik \textit{e-commerce} dan jasa kurir \cite{Ma2025}.

Berbagai pendekatan telah dikembangkan untuk menyelesaikan 3D-BPP. Pendekatan klasik meliputi algoritma heuristik seperti \textit{First Fit Decreasing} dan \textit{Best Fit} yang menempatkan item secara sekuensial berdasarkan aturan tertentu \cite{Silva2016}. Metode \textit{layer-building} membangun lapisan horizontal kemudian menumpuknya secara vertikal \cite{Tresca2022}. Pendekatan \textit{matheuristics} mengkombinasikan formulasi pemrograman linear dengan heuristik untuk mendapatkan solusi yang lebih baik dalam waktu komputasi yang wajar. Tresca et al. \cite{Tresca2022} mencatat bahwa sistem manajemen gudang komersial masih kekurangan algoritma yang efektif untuk mengotomatisasi konfigurasi palet secara optimal.

Perkembangan terkini menunjukkan penggunaan \textit{metaheuristic} dan kecerdasan buatan untuk 3D-BPP. Khairuddin et al. \cite{Khairuddin2020} mengembangkan simulator berbasis \textit{Genetic Algorithm} (GA) yang memvisualisasikan proses optimasi. Ma et al. \cite{Ma2025} mengkombinasikan \textit{block-building heuristic}, GA, dan \textit{simulated annealing} untuk menangani kargo heterogen. Ananno dan Ribeiro \cite{Ananno2024} mengusulkan algoritma dua tahap yang diuji dengan data industri nyata. Dari sisi kecerdasan buatan, \textit{Deep Reinforcement Learning} (DRL) mulai diterapkan untuk 3D-BPP. Zhao et al. \cite{Zhao2022} menggunakan DRL dengan \textit{stacking tree} untuk analisis stabilitas, sementara Murdivien dan Um \cite{Murdivien2023} memanfaatkan \textit{game engine} untuk simulasi pelatihan. Pendekatan hibrid heuristik-DRL juga menunjukkan hasil yang menjanjikan \cite{Wong2024}. Zhang et al. \cite{Zhang2024} mengusulkan kombinasi \textit{Generative Adversarial Network} dengan GA untuk meningkatkan kualitas solusi.

Aspek visualisasi berperan penting dalam solusi \textit{bin packing} karena memungkinkan validasi hasil dan pemberian panduan kepada operator. Krebs dan Ehmke \cite{Krebs2023} mengembangkan alat validasi dan visualisasi solusi secara \textit{open-source}, namun alat tersebut berbasis \textit{desktop} dan ditujukan untuk peneliti, bukan untuk panduan operasional di lapangan. Selain keterbatasan antarmuka, tantangan signifikan lainnya adalah pengujian algoritma yang seringkali hanya menggunakan dataset acak yang kurang mencerminkan skenario logistik dunia nyata \cite{Ribeiro2023}. Terdapat kesenjangan berupa kebutuhan akan sistem berbasis web yang mengintegrasikan algoritma \textit{bin packing} dengan visualisasi 3D interaktif yang mampu menangani skenario data yang realistis serta menyediakan panduan pemuatan tahap demi tahap.

Penelitian ini mengembangkan sistem perencanaan pemuatan kontainer berbasis web untuk menjembatani kesenjangan tersebut. Sistem memanfaatkan pustaka \textit{3D bin packing} yang menyediakan fitur pengecekan stabilitas dan simulasi gravitasi, kemudian mengintegrasikannya melalui arsitektur berlapis. Visualisasi 3D interaktif menampilkan hasil perhitungan algoritma dan menyediakan panduan pemuatan tahap demi tahap, sehingga operator dapat melihat urutan penempatan item secara visual. Sistem dapat diakses melalui \textit{browser} tanpa instalasi perangkat lunak khusus, sehingga meningkatkan aksesibilitas operasional.

Penelitian ini memiliki tiga tujuan utama. Pertama, mengintegrasikan pustaka \textit{3D bin packing} ke dalam sistem berbasis web. Kedua, mengembangkan visualisasi 3D interaktif dengan fitur \textit{playback} urutan pemuatan untuk panduan operator. Ketiga, mengevaluasi performa sistem melalui pengujian dengan skenario pemuatan yang realistis guna mengukur utilisasi ruang dan waktu komputasi yang dihasilkan.
