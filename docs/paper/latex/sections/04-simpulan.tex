% =============================================================================
% SECTION 4: SIMPULAN
% =============================================================================

\section{SIMPULAN}

Penelitian ini telah berhasil menyelesaikan pengembangan dan evaluasi sistem \textit{Load \& Stuffing Calculator} berbasis web yang dirancang untuk membantu optimasi pemuatan kargo heterogen. Melalui arsitektur berlapis yang memisahkan manajemen data dan mesin kalkulasi, sistem terbukti mampu mengintegrasikan algoritma \textit{3D bin packing} secara efektif ke dalam lingkungan berbasis web tanpa kendala operasional yang berarti. Hasil pengujian menunjukkan bahwa sistem mampu menangani skenario pemuatan hingga 300 unit barang dengan tingkat keberhasilan penempatan (\textit{fill rate}) mencapai 100\% secara konsisten pada seluruh skenario pengujian.

Dari sisi performa algoritma, penggunaan strategi \textit{Bigger First} yang dikombinasikan dengan batasan stabilitas statis dan simulasi gravitasi menghasilkan keterisian ruang yang optimal. Sistem mencatatkan utilisasi volume sebesar 55,26\% pada skenario kargo paling kompleks, sebuah angka yang kompetitif dan memenuhi standar efisiensi industri untuk penataan barang yang sangat beragam dimensi dan jenisnya. Meskipun waktu komputasi menunjukkan pola pertumbuhan kuadratik seiring bertambahnya populasi barang, durasi maksimal sekitar 38 detik untuk perhitungan 300 unit tetap berada dalam batas toleransi yang sangat memadai bagi kebutuhan perencanaan operasional gudang yang bersifat \textit{offline}.

Selain capaian kuantitatif, implementasi visualisasi tiga dimensi interaktif memberikan kontribusi penting dalam menjembatani kalkulasi teoritis dengan pelaksanaan fisik di lapangan. Fitur panduan langkah-demi-langkah (\textit{step playback}) memungkinkan operator untuk melihat urutan penempatan barang secara kronologis dengan mempertimbangkan aspek keamanan tumpukan melalui penopang minimal 75\% luas dasar barang. Secara keseluruhan, integrasi antara kalkulasi algoritma yang presisi dan antarmuka visual yang intuitif menjadikan sistem ini sebagai alat bantu pengambilan keputusan yang praktis untuk meningkatkan efisiensi proses logistik. Penelitian mendatang dapat diarahkan pada pengembangan strategi pengemasan yang lebih dinamis untuk berbagai tipe kontainer guna terus meningkatkan kepadatan muatan.
